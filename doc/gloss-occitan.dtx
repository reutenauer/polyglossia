% \iffalse
% !TEX encoding = UTF-8 Unicode
% !TEX TS-program = XeLaTeX
%<*internal>
\begingroup
\input docstrip.tex
\keepsilent

\preamble
  ------------------------------------------------------------------
  The gloss-occitan module for polyglossia
  Copyright (C) 2016 Cédric Valmary
  All rights reserved
  
  Licence information appended

  Created by Cédric Valmary: cvalmary at yahoo dot fr 
  of Tot en òc <http://www.totenoc.eu/>

\endpreamble
\postamble
Distributable under the LaTeX Project Public License,
version 1.3c or higher (your choice). The latest version of
this license is at: http://www.latex-project.org/lppl.txt

This work is "author-maintained"
The maintainer is Cédric Valmary

\endpostamble

\askforoverwritefalse

\generate{\file{gloss-occitan.ldf}{\from{gloss-occitan.dtx}{ldf}}}

\def\tmpa{plain}
\ifx\tmpa\fmtname\endgroup\expandafter\bye\fi
\endgroup
%</internal>
%
%<*driver>
\documentclass{ltxdoc}
\ProvidesFile{gloss-occitan.dtx}[2016/02/04 v.0.1 Documented Occitan module for polyglossia]
\usepackage[utf8]{inputenc}
\usepackage[LGR,T1]{fontenc}
\usepackage{lmodern}
\GetFileInfo{gloss-occitan.dtx} 
\title{The Occitan language module for polyglossia}
\date{\fileversion\space--- \filedate}
\author{Cédric Valmary --- \texttt{cvalmary at yahoo dot fr}}
\usepackage{metalogo,multicol,array,booktabs,graphicx}
\def\prog#1{\textsf{#1}}
\def\pack#1{\textsf{\slshape#1}}
\begin{document}\errorcontextlines=9
\maketitle
\begin{multicols}{2}
\tableofcontents
\end{multicols}
\setlength\hfuzz{20pt}
\DocInput{gloss-occitan.dtx}
\end{document}
%</driver>
%\fi
%
% \CheckSum{237}
%
%    \begin{abstract}
%     This file describes the Occitan language module for \pack{polyglossia}.
%     It describes also the options that may be specified and their 
%     functionalities.
%    \end{abstract}
%
%    \section{Usage}
%    When selecting the Occitan language with \pack{polyglossia} you have
%    to use either
%    \begin{flushleft}
%    \cs{setmainlanguage}\texttt{[babelshorthands]\{occitan\}}\\
%    or\\
%    \cs{setotherlanguage}\texttt{[babelshorthands]\{occitan\}}
%    \end{flushleft}
%    depending on the fact that Occitan is the main or a secondary document
%    language. The option |babelshorhands| is, in facts, optional; if
%    specified it defines the active double quote functionalities.
%    See table~\ref{tab:activedq}.
%    \begin{table}\centering
%    \begin{tabular}{>{\ttfamily}lp{0.7\textwidth}}
%    \toprule
%    |"|	& Followed by a single letter token inserts a compound 
%          word mark with the necessary discretionary break command and
%          allows hyphenation of both strings that precede and follow
%          this mark.\tabularnewline
%    \string"\string| & Behaves as |"| when the vertical bar is followed
%          by a complex token (a control sequence) or anything different
%          from a letter. \tabularnewline
%    |"<|	& Inserts open guillemets and eliminates space 
%          after «\tabularnewline
%    |">|	& Inserts closed guillemets and eliminates space 
%          before »\tabularnewline
%    |"/|	& Inserts a slash that allows hyphenation of both the preceding
%          and the following word.\tabularnewline
%    |".|	& Inserts a centerd dot (\emph{ponch interior}) with 
%          a discretionary break that allows hyphenation of both word 
%          fragments.\tabularnewline
%    \bottomrule
%    \end{tabular}
%    \caption{Occitan module shorthands}
%    \label{tab:activedq}
%    \end{table}
%A few words are in order.
%\begin{itemize}
%\item[|"|] This compound work marker is very useful. The hyphenation 
%    patterns developed for the Occitan language have been modelled on the 
%    French or Spanish ones especially for what concerns the separation 
%    between an `s' followed by another consonant. Nevertheless the 
%    \emph{Conselh de la lenga occitana} takes care of a common spelling 
%    for all the Occitan varieties, but leaves every speaker free to pronounce the 
%    way s/he likes best according to his/her dialect. Even if local 
%    pronunciations are allowed, the spelling varies a little between the 
%    Occitan varieties Auvernhat, Gascon, Lemosin, Lengadocian, Niçard, 
%    Provençal, and Vivaroalpenc. May be some particular spelling requires 
%    some manual intervention for a correct hyphenation. The hyphenation 
%    patterns have been developed taking into account some peculiarities 
%    of the Gascon and the Vivaroalpenc varieties, but better patterns 
%    will be created as long as suggestions/corrections arrive 
%    from the community of \LaTeX\ users.
%
%\item[\texttt{\string"\string|}] This shorthand should be useless 
%    within a \texttt{.tex} source file to be processed by UTF-8 
%    aware engines as \XeLaTeX\ and \LuaLaTeX. Nevertheless it might be 
%    necessary to insert a discretionary break in a strange word that 
%    requires a real macro within it; in this case the \verb+"|+ shorthand 
%    comes handy. The situation is different with 8-bit aware typesetting 
%    engines, because the \texttt{utf8} specified to the \pack{inputenc} 
%    package changes every non-\textsc{ascii} character into a LICR (LaTeX 
%    Internal Character Representation) which is substantially a macro; 
%    as such it is not recognised as a character by |"|, and this second 
%    compound word marker must be used.
%
%\item[|"<|] and |">| are used to set the guillemets with the proper 
%    spacing; French users generally leave in the source |.tex| file at
%    least one space after the open ones and another space before the
%    closed ones. This is supposed to be a bad practice  for what concerns
%    Occitan typesetting, therefore such commands take care of eliminating
%    these unwanted spaces, while simplifying the keying.
%    
%\item[|".|] This is a very special shorthand; it is intended to distinguish,
%    for example, `sh' from `s$\cdot$h' (and similarly for other such groups). 
%    For example \emph{dis$\cdot$har} is pronounced with a minimal pause 
%    between the sound of `s' and the aspired sound of `h'; without the centered 
%    dot (\emph{ponch interior}) the `sh' is a digraph that is pronounced as 
%    the IPA phoneme /\scalebox{0.8}{$\int$}/. When \emph{dis$\cdot$har}
%    (|dis".har| in the source file) gets hyphenated, it becomes
%    \mbox{\itshape dis-har}.
%\end{itemize}
%
% This ends the few peculiarities of the |gloss-occitan| language description file.
%
%^^A    \section*{Acknowledgements}
%^^A    When this file originated was very basic; Claudio Beccari added the
%^^A    double quote active features after the hyphenation patterns
%^^A    became available.
%
% \StopEventually{}
%
%    \section{Documented code}
%\iffalse
%<*ldf>
%\fi
%\subsection{Initial settings}
% First we have to identify this file, and we start with the initial code
% written by Cédric Valmary.
%    \begin{macrocode}
%************************************************* By Cédric Valmary
\ProvidesFile{gloss-occitan.ldf}[2016/02/04 v0.3 polyglossia: 
     module for Occitan]
%    \end{macrocode}
% Then we have ti set up \pack{polyglossia} in order to let the package 
% know what language is is handling; what is the name of the hyphenation
% pattern set; what are the minimum word fragment lengths of the first 
% and respectively the last word fragment before or after a line break; the
% specific setting for punctuation spacing, the indentation of the first 
%paragraph of a section; if \pack{polyglossia} should use a special font
% family |\occitanfont| in case the user defined such a family.
%    \begin{macrocode}
\PolyglossiaSetup{occitan}{
  hyphennames={occitan},
  hyphenmins={2,2},
  frenchspacing=true,
  indentfirst=true,
  fontsetup=true,
}
%************************************************
%    \end{macrocode}
%
% \subsection{Option definitions}
% We now document the contributed extension required to create the optional 
% functionality obtained from the double quote active character.
%
% We set up the necessary machinery for the module option
% |babelshorthands|. we set it as a boolean key thet does not require
% the explicit value |true| when it is specified to the module.
% The option must be tied to the Occitan language, so we also 
% define its prefix |occitan@|. We simultaneously use the switch
% |\ifsystem@baabelshorthands| in order to set the boolean key to |true|
% or |false|.
%    \begin{macrocode}
\define@boolkey{occitan}[occitan@]{babelshorthands}[true]{}

\ifsystem@babelshorthands
  \setkeys{occitan}{babelshorthands=true}
\else
  \setkeys{occitan}{babelshorthands=false}
\fi
%    \end{macrocode}
%
% At this point, in order to use the \pack{babel} machinery to define
% active characters, we test it if it was already loaded by testing
% the definiteness of a specific macro. If the module |babelsh.def|
% was not loaded, we load it, then start preparing the ground to
% define the double quote |"| as an active character.
%....\begin{macrocode}
\ifcsundef{initiate@active@char}{%
\ifx\initiate@active@char\@undefined
\else
  \bbl@afterfi\endinput
\fi
\ProvidesFile{babelsh.def}
         [2019/09/30 %
         Babel common definitions for shorthands^^J
         Taken verbatim from babel files (2019/09/27 v3.34)]
%
% ------------------------------------------------------------------------------
%
% lines 52 to 56 from babel.sty
%
% ------------------------------------------------------------------------------
%
\def\bbl@stripslash{\expandafter\@gobble\string}
\def\bbl@add#1#2{%
  \bbl@ifunset{\bbl@stripslash#1}%
    {\def#1{#2}}%
    {\expandafter\def\expandafter#1\expandafter{#1#2}}}
%
% ------------------------------------------------------------------------------
%
% line 73 to 74 from babel.sty
%
% ------------------------------------------------------------------------------
%
\long\def\bbl@afterelse#1\else#2\fi{\fi#1}
\long\def\bbl@afterfi#1\fi{\fi#1}
%
% ------------------------------------------------------------------------------
%
% line 399 from babel.sty
%
% ------------------------------------------------------------------------------
%
\let\bbl@opt@shorthands\@nnil
%
% ------------------------------------------------------------------------------
%
% lines 432 to 445 from babel.sty
%
% ------------------------------------------------------------------------------
%
\ifx\bbl@opt@shorthands\@nnil
  \def\bbl@ifshorthand#1#2#3{#2}%
\else\ifx\bbl@opt@shorthands\@empty
  \def\bbl@ifshorthand#1#2#3{#3}%
\else
  \def\bbl@ifshorthand#1{%
    \bbl@xin@{\string#1}{\bbl@opt@shorthands}%
    \ifin@
      \expandafter\@firstoftwo
    \else
      \expandafter\@secondoftwo
    \fi}
  \edef\bbl@opt@shorthands{%
    \expandafter\bbl@sh@string\bbl@opt@shorthands\@empty}%
%
% ------------------------------------------------------------------------------
%
% line 450 from babel.sty
%
% ------------------------------------------------------------------------------
%
\fi\fi
%
% ------------------------------------------------------------------------------
%
% lines 389 to 424 from switch.def
%
% ------------------------------------------------------------------------------
%
\ifx\PackageError\@undefined
  \def\bbl@error#1#2{%
    \begingroup
      \newlinechar=`\^^J
      \def\\{^^J(babel) }%
      \errhelp{#2}\errmessage{\\#1}%
    \endgroup}
  \def\bbl@warning#1{%
    \begingroup
      \newlinechar=`\^^J
      \def\\{^^J(polyglossia) }%
      \message{\\#1}%
    \endgroup}
  \def\bbl@info#1{%
    \begingroup
      \newlinechar=`\^^J
      \def\\{^^J}%
      \wlog{#1}%
    \endgroup}
\else
  \def\bbl@error#1#2{%
    \begingroup
      \def\\{\MessageBreak}%
      \PackageError{polyglossia}{#1}{#2}%
    \endgroup}
  \def\bbl@warning#1{%
    \begingroup
      \def\\{\MessageBreak}%
      \PackageWarning{polyglossia}{#1}%
    \endgroup}
  \def\bbl@info#1{%
    \begingroup
      \def\\{\MessageBreak}%
      \PackageInfo{polyglossia}{#1}%
    \endgroup}
\fi
%
% ------------------------------------------------------------------------------
%
% lines 48 to 69 from babel.def
%
% ------------------------------------------------------------------------------
%
\ifx\bbl@ifshorthand\@undefined
  \let\bbl@opt@shorthands\@nnil
  \def\bbl@ifshorthand#1#2#3{#2}%
  \let\bbl@language@opts\@empty
  \ifx\babeloptionstrings\@undefined
    \let\bbl@opt@strings\@nnil
  \else
    \let\bbl@opt@strings\babeloptionstrings
  \fi
  \def\BabelStringsDefault{generic}
  \def\bbl@tempa{normal}
  \ifx\babeloptionmath\bbl@tempa
    \def\bbl@mathnormal{\noexpand\textormath}
  \fi
  \def\AfterBabelLanguage#1#2{}
  \ifx\BabelModifiers\@undefined\let\BabelModifiers\relax\fi
  \let\bbl@afterlang\relax
  \def\bbl@opt@safe{BR}
  \ifx\@uclclist\@undefined\let\@uclclist\@empty\fi
  \ifx\bbl@trace\@undefined\def\bbl@trace#1{}\fi
  \expandafter\newif\csname ifbbl@single\endcsname
\fi
%
% ------------------------------------------------------------------------------
%
% line 108 from babel.def
%
% ------------------------------------------------------------------------------
%
\def\bbl@csarg#1#2{\expandafter#1\csname bbl@#2\endcsname}%
%
% ------------------------------------------------------------------------------
%
% lines 125 to 130 from babel.def
%
% ------------------------------------------------------------------------------
%
\def\bbl@exp#1{%
  \begingroup
    \let\\\noexpand
    \def\<##1>{\expandafter\noexpand\csname##1\endcsname}%
    \edef\bbl@exp@aux{\endgroup#1}%
  \bbl@exp@aux}
%
% ------------------------------------------------------------------------------
%
% lines 144 to 149 from babel.def
%
% ------------------------------------------------------------------------------
%
\def\bbl@ifunset#1{%
  \expandafter\ifx\csname#1\endcsname\relax
    \expandafter\@firstoftwo
  \else
    \expandafter\@secondoftwo
  \fi}
%
% ------------------------------------------------------------------------------
%
% lines 234 to 243 from babel.def
%
% ------------------------------------------------------------------------------
%
\chardef\bbl@engine=%
  \ifx\directlua\@undefined
    \ifx\XeTeXinputencoding\@undefined
      \z@
    \else
      \tw@
    \fi
  \else
    \@ne
  \fi
%
% ------------------------------------------------------------------------------
%
% lines 255 to 258 from babel.def
%
% ------------------------------------------------------------------------------
%
\def\bbl@withactive#1#2{%
  \begingroup
    \lccode`~=`#2\relax
    \lowercase{\endgroup#1~}}
%
% ------------------------------------------------------------------------------
%
% lines 443 to 558 from babel.def
%
% ------------------------------------------------------------------------------
%
\def\bbl@add@special#1{% 1:a macro like \", \?, etc.
  \bbl@add\dospecials{\do#1}% test @sanitize = \relax, for back. compat.
  \bbl@ifunset{@sanitize}{}{\bbl@add\@sanitize{\@makeother#1}}%
  \ifx\nfss@catcodes\@undefined\else % TODO - same for above
    \begingroup
      \catcode`#1\active
      \nfss@catcodes
      \ifnum\catcode`#1=\active
        \endgroup
        \bbl@add\nfss@catcodes{\@makeother#1}%
      \else
        \endgroup
      \fi
  \fi}
\def\bbl@remove@special#1{%
  \begingroup
    \def\x##1##2{\ifnum`#1=`##2\noexpand\@empty
                 \else\noexpand##1\noexpand##2\fi}%
    \def\do{\x\do}%
    \def\@makeother{\x\@makeother}%
  \edef\x{\endgroup
    \def\noexpand\dospecials{\dospecials}%
    \expandafter\ifx\csname @sanitize\endcsname\relax\else
      \def\noexpand\@sanitize{\@sanitize}%
    \fi}%
  \x}
\def\bbl@active@def#1#2#3#4{%
  \@namedef{#3#1}{%
    \expandafter\ifx\csname#2@sh@#1@\endcsname\relax
      \bbl@afterelse\bbl@sh@select#2#1{#3@arg#1}{#4#1}%
    \else
      \bbl@afterfi\csname#2@sh@#1@\endcsname
    \fi}%
  \long\@namedef{#3@arg#1}##1{%
    \expandafter\ifx\csname#2@sh@#1@\string##1@\endcsname\relax
      \bbl@afterelse\csname#4#1\endcsname##1%
    \else
      \bbl@afterfi\csname#2@sh@#1@\string##1@\endcsname
    \fi}}%
\def\initiate@active@char#1{%
  \bbl@ifunset{active@char\string#1}%
    {\bbl@withactive
      {\expandafter\@initiate@active@char\expandafter}#1\string#1#1}%
    {}}
\def\@initiate@active@char#1#2#3{%
  \bbl@csarg\edef{oricat@#2}{\catcode`#2=\the\catcode`#2\relax}%
  \ifx#1\@undefined
    \bbl@csarg\edef{oridef@#2}{\let\noexpand#1\noexpand\@undefined}%
  \else
    \bbl@csarg\let{oridef@@#2}#1%
    \bbl@csarg\edef{oridef@#2}{%
      \let\noexpand#1%
      \expandafter\noexpand\csname bbl@oridef@@#2\endcsname}%
  \fi
  \ifx#1#3\relax
    \expandafter\let\csname normal@char#2\endcsname#3%
  \else
    \bbl@info{Making #2 an active character}%
    \ifnum\mathcode`#2=\ifodd\bbl@engine"1000000 \else"8000 \fi
      \@namedef{normal@char#2}{%
        \textormath{#3}{\csname bbl@oridef@@#2\endcsname}}%
    \else
      \@namedef{normal@char#2}{#3}%
    \fi
    \bbl@restoreactive{#2}%
    \AtBeginDocument{%
      \catcode`#2\active
      \if@filesw
        \immediate\write\@mainaux{\catcode`\string#2\active}%
      \fi}%
    \expandafter\bbl@add@special\csname#2\endcsname
    \catcode`#2\active
  \fi
  \let\bbl@tempa\@firstoftwo
  \if\string^#2%
    \def\bbl@tempa{\noexpand\textormath}%
  \else
    \ifx\bbl@mathnormal\@undefined\else
      \let\bbl@tempa\bbl@mathnormal
    \fi
  \fi
  \expandafter\edef\csname active@char#2\endcsname{%
    \bbl@tempa
      {\noexpand\if@safe@actives
         \noexpand\expandafter
         \expandafter\noexpand\csname normal@char#2\endcsname
       \noexpand\else
         \noexpand\expandafter
         \expandafter\noexpand\csname bbl@doactive#2\endcsname
       \noexpand\fi}%
     {\expandafter\noexpand\csname normal@char#2\endcsname}}%
  \bbl@csarg\edef{doactive#2}{%
    \expandafter\noexpand\csname user@active#2\endcsname}%
  \bbl@csarg\edef{active@#2}{%
    \noexpand\active@prefix\noexpand#1%
    \expandafter\noexpand\csname active@char#2\endcsname}%
  \bbl@csarg\edef{normal@#2}{%
    \noexpand\active@prefix\noexpand#1%
    \expandafter\noexpand\csname normal@char#2\endcsname}%
  \expandafter\let\expandafter#1\csname bbl@normal@#2\endcsname
  \bbl@active@def#2\user@group{user@active}{language@active}%
  \bbl@active@def#2\language@group{language@active}{system@active}%
  \bbl@active@def#2\system@group{system@active}{normal@char}%
  \expandafter\edef\csname\user@group @sh@#2@@\endcsname
    {\expandafter\noexpand\csname normal@char#2\endcsname}%
  \expandafter\edef\csname\user@group @sh@#2@\string\protect@\endcsname
    {\expandafter\noexpand\csname user@active#2\endcsname}%
  \if\string'#2%
    \let\prim@s\bbl@prim@s
    \let\active@math@prime#1%
  \fi
  \bbl@usehooks{initiateactive}{{#1}{#2}{#3}}}
\@ifpackagewith{babel}{KeepShorthandsActive}%
  {\let\bbl@restoreactive\@gobble}%
  {\def\bbl@restoreactive#1{%
     \bbl@exp{%
%
% ------------------------------------------------------------------------------
%
% lines 561 to 755 from babel.def
%
% ------------------------------------------------------------------------------
%
       \\\AtEndOfPackage
         {\catcode`#1=\the\catcode`#1\relax}}}%
   \AtEndOfPackage{\let\bbl@restoreactive\@gobble}}
\def\bbl@sh@select#1#2{%
  \expandafter\ifx\csname#1@sh@#2@sel\endcsname\relax
    \bbl@afterelse\bbl@scndcs
  \else
    \bbl@afterfi\csname#1@sh@#2@sel\endcsname
  \fi}
\def\active@prefix#1{%
  \ifx\protect\@typeset@protect
  \else
    \ifx\protect\@unexpandable@protect
      \noexpand#1%
    \else
      \protect#1%
    \fi
    \expandafter\@gobble
  \fi}
\newif\if@safe@actives
\@safe@activesfalse
\def\bbl@restore@actives{\if@safe@actives\@safe@activesfalse\fi}
\def\bbl@activate#1{%
  \bbl@withactive{\expandafter\let\expandafter}#1%
    \csname bbl@active@\string#1\endcsname}
\def\bbl@deactivate#1{%
  \bbl@withactive{\expandafter\let\expandafter}#1%
    \csname bbl@normal@\string#1\endcsname}
\def\bbl@firstcs#1#2{\csname#1\endcsname}
\def\bbl@scndcs#1#2{\csname#2\endcsname}
\def\declare@shorthand#1#2{\@decl@short{#1}#2\@nil}
\def\@decl@short#1#2#3\@nil#4{%
  \def\bbl@tempa{#3}%
  \ifx\bbl@tempa\@empty
    \expandafter\let\csname #1@sh@\string#2@sel\endcsname\bbl@scndcs
    \bbl@ifunset{#1@sh@\string#2@}{}%
      {\def\bbl@tempa{#4}%
       \expandafter\ifx\csname#1@sh@\string#2@\endcsname\bbl@tempa
       \else
         \bbl@info
           {Redefining #1 shorthand \string#2\\%
            in language \CurrentOption}%
       \fi}%
    \@namedef{#1@sh@\string#2@}{#4}%
  \else
    \expandafter\let\csname #1@sh@\string#2@sel\endcsname\bbl@firstcs
    \bbl@ifunset{#1@sh@\string#2@\string#3@}{}%
      {\def\bbl@tempa{#4}%
       \expandafter\ifx\csname#1@sh@\string#2@\string#3@\endcsname\bbl@tempa
       \else
         \bbl@info
           {Redefining #1 shorthand \string#2\string#3\\%
            in language \CurrentOption}%
       \fi}%
    \@namedef{#1@sh@\string#2@\string#3@}{#4}%
  \fi}
\def\textormath{%
  \ifmmode
    \expandafter\@secondoftwo
  \else
    \expandafter\@firstoftwo
  \fi}
\def\user@group{user}
\def\language@group{english}
\def\system@group{system}
\def\useshorthands{%
  \@ifstar\bbl@usesh@s{\bbl@usesh@x{}}}
\def\bbl@usesh@s#1{%
  \bbl@usesh@x
    {\AddBabelHook{babel-sh-\string#1}{afterextras}{\bbl@activate{#1}}}%
    {#1}}
\def\bbl@usesh@x#1#2{%
  \bbl@ifshorthand{#2}%
    {\def\user@group{user}%
     \initiate@active@char{#2}%
     #1%
     \bbl@activate{#2}}%
    {\bbl@error
       {Cannot declare a shorthand turned off (\string#2)}
       {Sorry, but you cannot use shorthands which have been\\%
        turned off in the package options}}}
\def\user@language@group{user@\language@group}
\def\bbl@set@user@generic#1#2{%
  \bbl@ifunset{user@generic@active#1}%
    {\bbl@active@def#1\user@language@group{user@active}{user@generic@active}%
     \bbl@active@def#1\user@group{user@generic@active}{language@active}%
     \expandafter\edef\csname#2@sh@#1@@\endcsname{%
       \expandafter\noexpand\csname normal@char#1\endcsname}%
     \expandafter\edef\csname#2@sh@#1@\string\protect@\endcsname{%
       \expandafter\noexpand\csname user@active#1\endcsname}}%
  \@empty}
\newcommand\defineshorthand[3][user]{%
  \edef\bbl@tempa{\zap@space#1 \@empty}%
  \bbl@for\bbl@tempb\bbl@tempa{%
    \if*\expandafter\@car\bbl@tempb\@nil
      \edef\bbl@tempb{user@\expandafter\@gobble\bbl@tempb}%
      \@expandtwoargs
        \bbl@set@user@generic{\expandafter\string\@car#2\@nil}\bbl@tempb
    \fi
    \declare@shorthand{\bbl@tempb}{#2}{#3}}}
\def\languageshorthands#1{\def\language@group{#1}}
\def\aliasshorthand#1#2{%
  \bbl@ifshorthand{#2}%
    {\expandafter\ifx\csname active@char\string#2\endcsname\relax
       \ifx\document\@notprerr
         \@notshorthand{#2}%
       \else
         \initiate@active@char{#2}%
         \expandafter\let\csname active@char\string#2\expandafter\endcsname
           \csname active@char\string#1\endcsname
         \expandafter\let\csname normal@char\string#2\expandafter\endcsname
           \csname normal@char\string#1\endcsname
         \bbl@activate{#2}%
       \fi
     \fi}%
    {\bbl@error
       {Cannot declare a shorthand turned off (\string#2)}
       {Sorry, but you cannot use shorthands which have been\\%
        turned off in the package options}}}
\def\@notshorthand#1{%
  \bbl@error{%
    The character `\string #1' should be made a shorthand character;\\%
    add the command \string\useshorthands\string{#1\string} to
    the preamble.\\%
    I will ignore your instruction}%
   {You may proceed, but expect unexpected results}}
\newcommand*\shorthandon[1]{\bbl@switch@sh\@ne#1\@nnil}
\DeclareRobustCommand*\shorthandoff{%
  \@ifstar{\bbl@shorthandoff\tw@}{\bbl@shorthandoff\z@}}
\def\bbl@shorthandoff#1#2{\bbl@switch@sh#1#2\@nnil}
\def\bbl@switch@sh#1#2{%
  \ifx#2\@nnil\else
    \bbl@ifunset{bbl@active@\string#2}%
      {\bbl@error
         {I cannot switch `\string#2' on or off--not a shorthand}%
         {This character is not a shorthand. Maybe you made\\%
          a typing mistake? I will ignore your instruction}}%
      {\ifcase#1%
         \catcode`#212\relax
       \or
         \catcode`#2\active
       \or
         \csname bbl@oricat@\string#2\endcsname
         \csname bbl@oridef@\string#2\endcsname
       \fi}%
    \bbl@afterfi\bbl@switch@sh#1%
  \fi}
\def\babelshorthand{\active@prefix\babelshorthand\bbl@putsh}
\def\bbl@putsh#1{%
  \bbl@ifunset{bbl@active@\string#1}%
     {\bbl@putsh@i#1\@empty\@nnil}%
     {\csname bbl@active@\string#1\endcsname}}
\def\bbl@putsh@i#1#2\@nnil{%
  \csname\languagename @sh@\string#1@%
    \ifx\@empty#2\else\string#2@\fi\endcsname}
\ifx\bbl@opt@shorthands\@nnil\else
  \let\bbl@s@initiate@active@char\initiate@active@char
  \def\initiate@active@char#1{%
    \bbl@ifshorthand{#1}{\bbl@s@initiate@active@char{#1}}{}}
  \let\bbl@s@switch@sh\bbl@switch@sh
  \def\bbl@switch@sh#1#2{%
    \ifx#2\@nnil\else
      \bbl@afterfi
      \bbl@ifshorthand{#2}{\bbl@s@switch@sh#1{#2}}{\bbl@switch@sh#1}%
    \fi}
  \let\bbl@s@activate\bbl@activate
  \def\bbl@activate#1{%
    \bbl@ifshorthand{#1}{\bbl@s@activate{#1}}{}}
  \let\bbl@s@deactivate\bbl@deactivate
  \def\bbl@deactivate#1{%
    \bbl@ifshorthand{#1}{\bbl@s@deactivate{#1}}{}}
\fi
\newcommand\ifbabelshorthand[3]{\bbl@ifunset{bbl@active@\string#1}{#3}{#2}}
\def\bbl@prim@s{%
  \prime\futurelet\@let@token\bbl@pr@m@s}
\def\bbl@if@primes#1#2{%
  \ifx#1\@let@token
    \expandafter\@firstoftwo
  \else\ifx#2\@let@token
    \bbl@afterelse\expandafter\@firstoftwo
  \else
    \bbl@afterfi\expandafter\@secondoftwo
  \fi\fi}
\begingroup
  \catcode`\^=7  \catcode`\*=\active  \lccode`\*=`\^
  \catcode`\'=12 \catcode`\"=\active  \lccode`\"=`\'
  \lowercase{%
    \gdef\bbl@pr@m@s{%
      \bbl@if@primes"'%
        \pr@@@s
        {\bbl@if@primes*^\pr@@@t\egroup}}}
\endgroup
\initiate@active@char{~}
\declare@shorthand{system}{~}{\leavevmode\nobreak\ }
\bbl@activate{~}
%
% ------------------------------------------------------------------------------
%
% lines 890 to 927 from babel.def
%
% ------------------------------------------------------------------------------
%
\def\bbl@allowhyphens{\ifvmode\else\nobreak\hskip\z@skip\fi}
\def\bbl@t@one{T1}
\def\allowhyphens{\ifx\cf@encoding\bbl@t@one\else\bbl@allowhyphens\fi}
\newcommand\babelnullhyphen{\char\hyphenchar\font}
\def\babelhyphen{\active@prefix\babelhyphen\bbl@hyphen}
\def\bbl@hyphen{%
  \@ifstar{\bbl@hyphen@i @}{\bbl@hyphen@i\@empty}}
\def\bbl@hyphen@i#1#2{%
  \bbl@ifunset{bbl@hy@#1#2\@empty}%
    {\csname bbl@#1usehyphen\endcsname{\discretionary{#2}{}{#2}}}%
    {\csname bbl@hy@#1#2\@empty\endcsname}}
\def\bbl@usehyphen#1{%
  \leavevmode
  \ifdim\lastskip>\z@\mbox{#1}\else\nobreak#1\fi
  \nobreak\hskip\z@skip}
\def\bbl@@usehyphen#1{%
  \leavevmode\ifdim\lastskip>\z@\mbox{#1}\else#1\fi}
\def\bbl@hyphenchar{%
  \ifnum\hyphenchar\font=\m@ne
    \babelnullhyphen
  \else
    \char\hyphenchar\font
  \fi}
\def\bbl@hy@soft{\bbl@usehyphen{\discretionary{\bbl@hyphenchar}{}{}}}
\def\bbl@hy@@soft{\bbl@@usehyphen{\discretionary{\bbl@hyphenchar}{}{}}}
\def\bbl@hy@hard{\bbl@usehyphen\bbl@hyphenchar}
\def\bbl@hy@@hard{\bbl@@usehyphen\bbl@hyphenchar}
\def\bbl@hy@nobreak{\bbl@usehyphen{\mbox{\bbl@hyphenchar}}}
\def\bbl@hy@@nobreak{\mbox{\bbl@hyphenchar}}
\def\bbl@hy@repeat{%
  \bbl@usehyphen{%
    \discretionary{\bbl@hyphenchar}{\bbl@hyphenchar}{\bbl@hyphenchar}}}
\def\bbl@hy@@repeat{%
  \bbl@@usehyphen{%
    \discretionary{\bbl@hyphenchar}{\bbl@hyphenchar}{\bbl@hyphenchar}}}
\def\bbl@hy@empty{\hskip\z@skip}
\def\bbl@hy@@empty{\discretionary{}{}{}}
\def\bbl@disc#1#2{\nobreak\discretionary{#2-}{}{#1}\bbl@allowhyphens}
%
% ------------------------------------------------------------------------------
%
% end of the code copied from babel files
%
% ------------------------------------------------------------------------------
%
\def\bbl@disc@german#1#2{%
  \nobreak\discretionary{#2-}{}{#1}}
\endinput
%
\initiate@active@char{"}%
}{}
%    \end{macrocode}
%
% Now we are ready to assign a definition to the double quote |"| active
% character. The |"| active char is supposed to
% do a certain small collection of actions, different in math mode
% compared to text mode; therefore we define a service macro |\xpgoc@next| with a
% different meaning depending on the typesetting mode. Notice that in
% text mode the definition assigns to a token the meaning of the token
% that upon expansion of the macro follows directly |\xpgoc@cwm|.
% The assignment with |\futurelet| is executed before |\xpgoc@cwm|
% therefore it can pick up also the first space token that possibly
% follows the expansion of |"|; a macro would ignore such space.
%    \begin{macrocode}
\def\occitan@shorthands{%
  \bbl@activate{"}%
  \def\language@group{occitan}%
  \declare@shorthand{occitan}{"}{%
    \relax\ifmmode
      \def\xpgoc@next{''}%
    \else
      \def\xpgoc@next{\futurelet\xpgoc@temp\xpgoc@cwm}%
    \fi
  \xpgoc@next}%
}
%    \end{macrocode}
%
% \subsection{The double quote active character}
%
% We now define a couple of service macros; |\xpgoc@@cwm| expands to
% an absolute |nobreak| macro that forbids any line break; then a
% normal discretionary (the long definition with three arguments,
% is made through a primitive command, but if we used the standard
% |\-| control character, we would get the same performance); finally
% we put another |\nobreak| command and a zero width glob of glue;
% this zero-width, zero-stretch, zero-shrink glob of glue does not
% interfere with typesetting but is the actual trick that lets the
% typesetting engine understand that the incoming string of letters
% has to be treated as a word, so that the hyphenation algorithm
% continues working after the discretionary break.
% 
% Similarly the macro |\xpgoc@ponchinterior| works in the same way,
% but the discretionary break contains a non empty third argument
% that contains a box which in turn contains the centered dot.
%    \begin{macrocode}
\def\xpgoc@@cwm{\nobreak\discretionary{-}{}{}\nobreak\hskip\z@skip}
\def\xpgoc@ponchinterior{%
      \nobreak\discretionary{-}{}{\mbox{$\cdot$}}\nobreak\hskip\z@skip}
%    \end{macrocode}
%
% Eventually we can make a chain of conditional statements that check
% what the |\xpgoc@temp| let token actually represents. We define
% another service macro |\xpgoc@@next| macro with different definitions
% depending on the status of the various tests. We first let it to a
% |\relax| no-op, in case that none of the tests is true; eventually
% we use its expansion that will do different actions depending on the
% tests. Notice that all the used conditionals expand their arguments;
% therefore it is necessary to use |\noexpand| in order to maintain the
% integrity of the tokens to be compared. Notice also that the service
% macro is sometimes defined as an argument-less macro, and sometimes
% as a macro with one compulsory argument; in this latter case, since
% we are making definitions within another definition we have to double
% the hash sign. In the latter case it will ignore any spaces following
% it and get the first non blank token; in most cases it will gobble the
% first non blank token and discard it.
%    \begin{macrocode}
\def\xpgoc@cwm{\let\xpgoc@@next\relax
  \ifcat\noexpand\xpgoc@temp a%
                         \def\xpgoc@@next{\xpgoc@@cwm}%
  \else
    \if\noexpand\xpgoc@temp \string|%
      \def\xpgoc@@next##1{\xpgoc@@cwm}%
    \else
      \if\noexpand\xpgoc@temp \string<%
        \def\xpgoc@@next##1{«\ignorespaces}%
      \else
        \if\noexpand\xpgoc@temp \string>%
          \def\xpgoc@@next##1{\unskip»}%
        \else
          \if\noexpand\xpgoc@temp\string/%
            \def\xpgoc@@next##1{\slash}%
          \else
            \if\noexpand\xpgoc@temp\string.%
              \def\xpgoc@@next##1{\xpgoc@ponchinterior}%
            \fi
          \fi
        \fi
      \fi
    \fi
  \fi
  \xpgoc@@next}
%    \end{macrocode}
%
% Before going on we have to define what to delete when leaving the Occitan
% typesetting, so that another language may start working without any
% residue of the Occitan settings. In particular the double quote |"| active
% char must be deactivated.
%    \begin{macrocode}
\def\nooccitan@shorthands{%
  \@ifundefined{initiate@active@char}{}{\bbl@deactivate{"}}%
}
%    \end{macrocode}
%
% \subsection{Occitan infix words}
% We can resume the original code for defining the infix Occitan
% words and to typeset the date.
%    \begin{macrocode}
%*********************************************** By Cédric Valmary
\def\captionsoccitan{%
   \def\refname{Referéncias}%
   \def\abstractname{Resumit}%
   \def\bibname{Bibliografia}%
   \def\prefacename{Prefaci}%
   \def\chaptername{Capítol}%
   \def\appendixname{Annèx}%
   \def\contentsname{Ensenhador}%
   \def\listfigurename{Taula de las figuras}%
   \def\listtablename{Taula dels tablèus}%
   \def\indexname{Indèx}%
   \def\figurename{Figura}%
   \def\tablename{Tablèu}%
   %\def\thepart{}%
   \def\partname{Partida}%
   \def\pagename{Pagina}%
   \def\seename{vejatz}%
   \def\alsoname{vejatz tanben}%
   \def\enclname{Pèça junta}%
   \def\ccname{còpia a}%
   \def\headtoname{A}%
   \def\proofname{Demostracion}%
   \def\glossaryname{Glossari}%
}
\def\dateoccitan{%
   \def\occitanmonth{\ifcase\month\or
      de~genièr\or
      de~febrièr\or
      de~març\or
      d'abril\or
      de~mai\or
      de~junh\or
      de~julhet\or
      d'agost\or
      de~setembre\or
      d'octobre\or
      de~novembre\or
      de~decembre\fi
   }%
   \def\occitanday{\ifcase\day\or
      1èr\else% primièr
      \number\day\fi% all other numbers
   }%
   \def\today{\occitanday\space \occitanmonth\space de~\number\year}%
}
%*************************************************
%    \end{macrocode}
%
% \subsection{Final clean-up}
% \pack{polyglossia} requires that  at |\begin{document}| time certain
% values are saved. Actually this is a prudential action: the user might
% have changed the normal values, so before setting the values valid
% for Occitan typesetting, its better to save the general values.
% The actual values that require saving are the club and widow penalties,
% together with the final demerits; the latter ones are the demerits in
% hyphenating the penultimate line of a paragraph so as not to terminate
% with a last line composed with a single syllable. The internal value
% |\@clubpenalty| must be saved, because sometimes it does not equal
% that of |\clubpenalty|
%    \begin{macrocode}
\let\xpgoc@savedvalues\empty
\AtEndPreamble{% the user or the class might define different values
  \edef\xpgoc@savedvalues{%
    \clubpenalty=\the\clubpenalty\space
    \@clubpenalty=\the\@clubpenalty\space
    \widowpenalty=\the\widowpenalty\space
    \finalhyphendemerits=\the\finalhyphendemerits}
}
%    \end{macrocode}
%
% Eventually we define the definitive |\noextras@occitan| macro to
% undo everything that was done for setting up the typesetting of
% the Occitan language.
%    \begin{macrocode}
\def\noextras@occitan{%
   \lccode\string"2019=\z@
   \nooccitan@shorthands
   \xpgoc@savedvalues
}
%    \end{macrocode}
%
% For setting up Occitan typesetting \pack{polyglossia} requires two different
% settings: the general ones and the specific settings for typesetting in line.
%    \begin{macrocode}
\def\blockextras@occitan{%
   \lccode\string"2019=\string"2019
   \clubpenalty=3000 \@clubpenalty=3000 \widowpenalty=3000
   \finalhyphendemerits=50000000
   \ifoccitan@babelshorthands\occitan@shorthands\fi
}

\def\inlineextras@occitan{%
   \lccode\string"2019=\string"2019
   \ifoccitan@babelshorthands\occitan@shorthands\fi
}
%    \end{macrocode}
%
% This ends the documentation of the Occitan language module for
% \pack{polyglossia}.
%\iffalse
%</ldf>
%\fi
%
% \Finale
% \endinput
%
