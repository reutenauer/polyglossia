\setotherlanguage[calendar=gregorian,numerals=western]{urdu}
\newfontfamily\urdufont[Script=Arabic,Scale=1.7]{Scheherazade}

\begin{urdu}
اُردو ایک ہند- آریائی زبان ہے جس کا تعلق ہند-یورپی لسانی خاندان کی ہند-ایرانی شاخ سے ہے.  جنوبی ایشیا میں سلطنتِ دہلی اور مغلیہ سلطنت کے دوران ہندوستانی کی مقامی زبانوں اور فارسی، عربی، اور تُرکی زبانوں کے اختلاط سے اردو کا ارتقا ہوا.\\

اُردو (بولنے والوں کی تعداد کے لحاظ سے) دُنیا کی تمام زبانوں میں بیسویں نمبر پر ہے. یہ پاکستان کی قومی زبان ہے اور بھارت کی 23 سرکاری زبانوں میں سے ایک ہے. تاریخی تعلقات اور پاکستان میں افغانی مہاجریں کی بڑی آبادی کے سبب سے اکثر افغانی بھی اردو سمجھتے، بولتے، اور لکھتے ہیں۔ جنوبی ایشیاء سے باہر اُردو زبان خلجِ فارس کے ممالک، سعودی عرب. برطانیہ، امریکہ، کینیڈا، جرمنی، ناروے اور آسٹریلیاء میں مقیم جنوبی ایشیائی مہاجرین بولتے ہیں۔\\

اُردو کا بعض اوقات ہندی کے ساتھ موازنہ کیا جاتا ہے. ان دونوں زبانوں کی قواعد ایک ہے۔ اور بنیادی ذخیرۂِ الفاظ بڑی حد تک مشترک ہے۔ لیکن مشکل خیالات کے اظہار کے لئے اُردو کے الفاط زیادہ تر فارسی اور عربی سے اخذ ہوئے ہیں اور ہندی کے الفاظ سنسکرت سے۔ اس کے علاوہ ایک بڑا فرق یہ ہے کہ اردو عربی رسم الخط میں لکھی جاتی ہے جبکہ ہندی دیوناگری رسم الخط میں۔ کچھ ماہرینِ لسانیات اُردو اور ہندی کو ایک ہی زبان کی دو معیاری صورتیں گردانتے ہیں. اگر اردو اور ہندی کو ملا کر دیکھا جائے تو پھر یہ دُنیا میں سب سے زیادہ بولے جانی والی چوتھی زبان ہے۔\\

(\today\ = \Hijritoday[0])
\end{urdu}

