\documentclass{article}
\usepackage{polyglossia}

\setmainlanguage{uyghur}
\newfontfamily\arabicfont{UKIJ Tuz}
\let\arabicfonttt\ttfamily

\title{سالام ئۇيغۇرلار بۇ polyglossia}
\author{كاۋىچى}
\date{\today}
\begin{document}
\maketitle

بۈگۈن(\date{\today}) نورۇز بايرىمى كۈنىدۇر. نورۇز بايرىمىڭلار مۇبارەك بولسۇن.


«پەش ۋەقەسى»

ئۇيغۇر ئەدەبىياتىنىڭ نەمۇنىلىك شەخسلىرىدىن لۇتپۇللا مۇتەللىپ بىلەن ئابدۇرېھىم ئۆتكۈرنىڭ سەبداشلىقى ، ئۆز ئەسەرلىرىدە بىر-بىرىگە تەسىر كۆرسەتكەنلىكى ۋە بىرلىكتە «چىن مودەن » دىراممىسىنى يازغانلىقى قېرىنداشلارغا ئانچە ناتونۇش بولمىسا كېرەك. ئابدۇرېھىم ئۆتكۈر ل.مۇتەللىپكە «قاينام» دېگەن تەخەللۇسنى تەقدىم قىلغان، كېيىن ل.مۇتەللىپ كەينىگە «ئۆركەش» سۆزىنى قوشۇپ «قاينام ئۆركىشى» دەپ قوللانغان. ل. مۇتەللىپمۇ ئا.ئۆتكۈرگە «ئۆتكۈر» دېگەن تەخەللۇسنى تەقدىم قىلغان . ئا.ئۆتكۈر 1941-يىلىدىن كېيىن ئەسەرلىرىدە «ئۆتكۈر» دېگەن تەخەللۇسنى قوللانغان.

ئوبزورچى ئەنۋەر ئابدۇرېھىم بىلەن ئابدۇرېھىم ئۆتكۈرنىڭ سۆھبەت خاتىرىسىگە ئاساسەن ، ل.مۇتەللىپ بىلەن ئا. ئۆتكۈرنىڭ دوستلۇقى ھەققىدىكى «پەش ۋەقەسى» نى قېرىنداشلارغا سۇنماقچىمەن:

1943-يىلنىڭ 11-ئېيى بولسا كېرەك، لۇتپۇللا مۇتەللىپ ئەينى چاغدىكى مەركىزىي ئۇيغۇر ئۇيۇشمىسىدا ئابلېتىپ راخمان (دىلشات) باشچىلىقىدا مىللىي سەنئەتكە نىسبەتەن قىلىنىۋاتقان بۇزغۇنچىلىققا چىداپ تۇرالماي، ئۇيغۇر مىللىي سەنئىتىنى قوغداش ۋە ئۇنىڭ سەھنىدىكى ناماياندىلىرى بولغان مەرھۇم قاسىمجان قەمبىرى، مەرۇپ ئەپەندى، سىراجىدىن زۇپەر، قەمبەر خانىم قاتارلىق سەنئەتكارلارغا مەدەت بېرىش يۈزىسىدىن ئاتالمىش «شىنجاڭ گېزىتى» دە «سەنئەت خۇمارى» تەخەللۇسى بىلەن «سەنئەتكە مۇھەببەت» ناملىق ماقالىسىنى ئېلان قىلىدۇ. ئابلېتىپ راخمان (دىلشات) بۇنىڭغا قارشى «خالىس» دېگەن تەخەللۇس بىلەن ل. مۇتەللىپكە ھۇجۇم قىلىدۇ. مۇتلەق كۆپچىلىك ئىلغار زىيالىيلار لۇتپۇللا تەرەپتە تۇرۇپ، كەسكىن ۋە ھەققانىي جامائەت پىكرىنى مەيدانغا كەلتۈرىدۇ. ئابدۇرېھىم ئۆتكۈرمۇ «سەنئەتتە مىللىيلىك» دېگەن تېمىدا بىر ماقالا يېزىپ ، لۇتپۇللا مۇتەللىپنى ياقلاپ چىقىدۇ. شۇنداق قىلىپ، ئابلېتىپ راخمان بىلەن بولغان زىددىيەت تېخىمۇ كۈچىيىدۇ. ل.مۇتەللىپنىڭ ئاقسۇدا يازغان «شائىر توغرىسىدا مۇۋەششەھ» ناملىق شېئىرى ئاشۇ كۈرەشنىڭ ئىنكاسى ئىدى. ئۇشبۇ شېئىرنىڭ ئاخىرقى مىسراسىدىكى «خالىس» دېگەن سۆز ئاشۇ ئابلېتىپ راخمان(دىلشات) غا قارىتىپ ئېيتىلغان ئىدى. لېكىن كېيىن بەزى كىشىلەر بۇ مىسرادىكى "پەش" نى غەرەزلىك ھالدا يۆتكەپ، «سايرا ئۆتكۈر، ھاڭ-تاڭ قالسۇن ، گۈل-گۈلىستان بول» دېگەن مىسرانىڭ ئورنىغا «سايرا، ئۆتكۈر ھاڭ-تاڭ قالسۇن، گۈل-گۈلىستان بول» دېگەن شەكىلدە ئېلان قىلىپ ، ھەر خىل چۈشەنچىلەرنى پايدا قىلىپ بۇ ئىككى زاتنىڭ ئارىسىغا زىددىيەت ئۇرۇقى چاچماقچى بولغان. بەختكە يارىشا، 1981- يىلى پىشقەدەم يازغۇچىلاردىن ئەلقەم ئەختەم ئۆزىنىڭ «ل. مۇتەللىپ ئەسەرلىرى» ناملىق كىتابىدا بۇ مەسىلىگە چۈشەنچە بېرىپ، ھەقىقىي ئەھۋالنى بايان قىلىپ، قىرىق يىللىق ئاھانەتكە تۈزىتىش بېرىدۇ. شۇنىڭدىن كېيىن ، بۇ مەسىلە ئايدىڭلىشىدۇ. ئەسلىدە ل.مۇتەللىپ بۇ شېئىرىنى ئاقسۇدىن «شىنجاڭ گېزىتى» گە ئەۋەتكەندە، يەنە ئايلىنىپ كېلىپ ، ئابلېتىپ راخماننىڭ قولىغا چۈشكەن ۋە ئۇنىڭ توسقۇنلۇقى بىلەن ئۆزگەرتىلىپ ئېلان قىلىنغان.

نەۋباھار تەييارلىدى

\end{document}
